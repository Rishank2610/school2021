\section{Git Hosting Providers}
\headlineframe{Git Hosting Providers}

\begin{frame}[c]{Git Hosting Providers}
  \begin{itemize}
    \item Several Providers and self-hosted server solutions are available
    \item Usually provide much more than just hosting the repositories
      \begin{itemize}
        \item Issue tracking
        \item Code review using pull requests
        \item Wiki
        \item Project Management, e.g. Canban boards
        \item Continuous integration
        \item Releases
      \end{itemize}
  \end{itemize}
  
\end{frame}

\begin{frame}[c]{Git Hosting Providers}
  \small
  \begin{tabu}{@{} X[,C] @{} X[,C] @{} X[,C] @{}}
    \href{https://github.com}{\includegraphics[width=0.75\linewidth]{figures/github.png}} &
    \href{https://gitlab.com}{\includegraphics[width=0.75\linewidth]{figures/gitlab.png}} &
    \href{https://bitbucket.org}{\includegraphics[width=0.75\linewidth]{figures/bitbucket.png}} \\
    \begin{itemize}
      \item Largest Hoster
      \item Many Open Source Projects, e.g. Python
      \item Unlimited private repositories
      \item Free CI service for public repositories
      \item Github Pro free for students / teachers / researchers
    \end{itemize}
    &
    \begin{itemize}
      \item open-source community edition
      \item paid enterprise edition with more features
      \item unlimited private repositories
      \item Self hosted or as service at \href{https://gitlab.com}{gitlab.com}
    \end{itemize}
    &
    \begin{itemize}
      \item Unlimited private repos with up to 5 contributors
      \item Lacks far behind GitHub and GitLab
    \end{itemize}
  \end{tabu}
\end{frame}

\begin{frame}{SSH Keys}
  Git can communicate using two ways with a remote:
  \begin{description}
    \item[HTTP] Works out of the box, requires entering credentials at every push/pull
    \item[SSH] Using private/public keys, you only need to enter the key password once per session
  \end{description}

  SSH-Keys:
  \begin{enumerate}
    \item \texttt{ssh-keygen -t ed25519 -C "GitHub Key for <username> at <machine>" -f ~/.ssh/id\_rsa.github}
    \item Choose a password
    \item \texttt{cat \textasciitilde/.ssh/id\_rsa.github.pub}
    \item Add key to profile
  \end{enumerate}
\end{frame}

\begin{frame}[c]{Pull Requests}
  
\end{frame}


\begin{frame}[c]{Forking}
  \begin{itemize}
    \item Using git and hosting providers, it's easy to contribute to projects you do not have write access to.

    \item This is arguably the most important reason for git's success.

    \item Forking means to create a copy of the main repository in your namespace, e.g. \url{http://github.com/matplotlib/matplotlib} to \texttt{http://github.com/maxnoe/matplotlib}

    \item You can then make changes and create a pull request in the main repository!

    \item To keep your fork up to date, you should add both your fork and the main repo as remotes.
  \end{itemize}
\end{frame}

\begin{frame}{Forks}
  \begin{tabu}{>{\ttfamily}l X[,L]}
    git clone <your fork>  & Clone your fork \\
    git add remote upstream <main repo>   & Add the main repo \\
    git fetch upstream   & Download changes from the main repo \\
    git reset --hard upstream/master & Reset the current branch to the master of the main repo to synchronize with the changes
  \end{tabu}
\end{frame}

\begin{frame}[c]{Issue Tracking}
\end{frame}

\begin{frame}[c, fragile]{Commit Integration with Issue Tracking}
  Start working on fixing a bug, that was documented in issue 42.

  \begin{code}{text}
  $ git checkout -b fix_42

  ... do stuff to fix bug ...

  $ git add src/foo.cxx
  $ git commit -m "Fix segmentation fault when doing stuff, fixes #42" 
  $ git push -u origin fix_42
  \end{code}

  If this commit get's merged into master, issue 42 will automatically be closed.
\end{frame}

\begin{frame}[c]{Code Review}
\end{frame}

\begin{frame}[c]{Continuous Integration}
\end{frame}
